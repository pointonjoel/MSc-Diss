\addcontentsline{toc}{chapter}{Abstract}

\begin{abstract} %200-500 words

%An abstract (200-500 words) that summarises what the project is about, what it has achieved, and how it contributes to the development of research and/or technology in the area. This should also include  5 to 10 keywords you think would help someone trying to find your dissertation (e.g., in a web search). Please be careful to enter specific keywords relevant to your dissertation, don’t be too general. We recommend that you include full names and acronyms where appropriate, and separate key word with semi-colons e.g. "Keywords: Human Computer Interaction (HCI); Internet of Things (IoT); autonomous vehicles; user study; qualitative study"

The recently released ChatGPT stunned users with its human-sounding content and ability to assist with a wide range of tasks. However, \acrlong{llm}s (\acrshort{llm}s) such as \acrshort{gpt} generate content in a probabilistic manner and so, at present, cannot guarantee factual answers or provide sources. This paper presents a framework to achieve this, by using a document store to pass relevant information to a chatbot which has been fine-tuned for question-answering. The document store is generated automatically from chosen documents which are split into chunks of 800 tokens.

This paper takes a novel approach to creating a question-answering framework. By adapting existing summarisation fine-tuning techniques, and enhancing Google's long-form Natural Questions dataset using \acrshort{gpt}-3.5 Turbo, a model is fine-tuned to create a generative (and not extractive) question-answering chatbot. Further, this paper contributes a long-form, generative answer variant of the Natural Questions dataset to the literature.

The proposed framework and provided model can, with future development, be used in countless applications. The long-form, closed-domain question-answering chatbot created by this paper can be used to answer questions from any domain, provided that context is given from a document store. Therefore, future adaptations to complete the model could enable it to be used in production, to benefit academics and students by streamlining \acrshort{qa} processes. Other applications such as customer service, legal advice, and other applications are also possible, though not explored.

The model achieves excellent results on the validation set: \acrshort{rouge} scores of between 43 and 45, a \acrshort{bleu} score of 34.2, and a \acrshort{meteor} score of 0.405. Validation answers attain an average cosine similarity of 0.763, with a 10.4\% false negative rate and 30.2\% false positive rate.

All code can be found in the following GitHub repository: \url{https://github.com/pointonjoel/MSc-Diss}, and the proposed model can be found in the following HuggingFace repository: \url{https://huggingface.co/psxjp5/mt5-small}.

\textbf{Keywords}: \acrshort{llm}; \acrfull{nlp}; \acrfull{gpt}; \acrfull{mlm}; Fine-Tuning

\end{abstract}