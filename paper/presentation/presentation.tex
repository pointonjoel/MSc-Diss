\documentclass[aspectratio=169,12pt]{beamer} 
\usepackage[utf8]{inputenc}
\usepackage{../beamer_preamble}
\usepackage{import}
% \setbeamertemplate{section in toc}[sections numbered]
% \setbeamertemplate{subsection in toc}[subsections numbered]

% 12 minute (2-3 minutes for questions)
% Your talk should cover:
% •	The aim(s) of your project 
% •	What has been achieved (including significance and contribution)
% •	How you achieved these outcomes 

% Keep it simple/big picture, don't dive into the technical details too much and ensure someone can understand it without reading your presentation.

\begin{document}
%Title----------------------------------------------
\section{Introduction}
\title{The Use of AI in Education}
\subtitle{Main findings}
\author[Pointon, Joel]{J.~Pointon\\\footnotesize{\textit{Supervised by T.~Chandesa}}\\ \vspace{20pt}\includegraphics[width=2.3cm]{../images/nottingham-logo.png}}

\date[September 2023] % (optional)
{September 2023}

\frame{\titlepage}

\addtobeamertemplate{frametitle}{}{%
\begin{textblock*}{110mm}(0.935\textwidth,0.91cm)
\includegraphics[height=30.5pt]{../images/nottingham-logo.png}
\end{textblock*}}

%TOC----------------------------------------------
\begin{frame}
\frametitle{Overview} 
% \tableofcontents
\tableofcontents
\end{frame}

%2 - Theory----------------------------------------------
\section{The Economics of Discrimination}
\begin{frame}[t]{The Economics of Discrimination}
\onslide<1->{
		\begin{itemize}
	    	\item Equal participation and progression could increase GDP by £24bn gains per year - 1.3\% of GDP (GOVg, 2017)
		    \item Moral and economic problem affecting individuals
			\item A lack of ethnicity research, particularly recently and across time (Heath and Cheung (2006) and Brynin and Güveli (2012))
		\end{itemize}
}
\onslide<2>{
		\begin{enumerate}
		    \item Individual/direct pay discrimination (Human capital model suggests it is unlikely)
			\item Institutional discrimination (statistical discrimination)
			\item Systemic discrimination (Not considered in this paper)
		\end{enumerate}
}
\end{frame}

%3 - Literature----------------------------------------------
\begin{frame}[t]{Literature}
\onslide<1->{
		\begin{itemize}
		    \item Higher pay gap for those born overseas (Berthoud, 2000)
            \item Chinese, Indian and mixed workers receive higher annual pay than White British workers (ONS, 2018)
		    \item Chinese differential is often attributed to a relative drive for educations (Leslie and Drinkwatr, 1999)
            \item Mix of endownments and \enquote*{unexplained} factors (Blackaby et al., 2005)
            \item Statistically significant levels of discrimination (Metcalf, 2009)
			\item Occupational clustering is a key factor (Brynin and Güveli, 2012)
			\item Lack of analysis of female ethnic wage gap
		\end{itemize} %lack of recent research and over time
}
\end{frame}

%4 - Methodology----------------------------------------------
\section{Methodology}
\begin{frame}[t]{Methodology}
\onslide<1->{
			\begin{itemize}
				\item LFS, combined with US and ONS data
				\item Merged 1997-2019 inclusive, using comparable variables
				\item  497,144 White (93.3\%), 10,096 Black (1.9\%), 1,874 Chinese (0.4\%), 20,265 who are Asian but not Chinese (3.8\%), and 3,585 mixed race (0.7\%)
			\end{itemize}
}
\onslide<2>{
    \begin{itemize}
		\item Blinder-Oaxaca decompositions
    \begin{equation}
      D = [\bar{x_{w}} - \bar{x_{b}}]\beta_{b} + [\beta_{w} - \beta_{b}]\bar{x_{b}}
      \label{eq:1}
    \end{equation}
    \vspace{-15pt}
    \begin{equation}
      D = [Explained] + [Unexplained]
      \label{eq:2}
    \end{equation}	
    \item Unexplained part includes direct pay discrimination
    \end{itemize}
}

\end{frame}

%5/6 - Descriptive Analysis-------------------------------------------
\begin{frame}{Descriptive Analysis: Raw Pay Gap}
\onslide<1>{
%\vspace{-10pt}
    \begin{columns}
    \column{0.58\textwidth}
        \vspace{5pt}
		% \import{beamer/}{median_wages}
	\column{0.42\textwidth}
	    \vspace{-40pt}
    	\begin{itemize}
    	    \item Low sample size increases volatility
    	    \item Asian workers consistently paid less
    	    \item Deterioration for Black workers post-financial crisis
    	    \item Chinese and mixed race workers outperformed wages of Whites
    	\end{itemize}   
	\end{columns}
	%\vspace{-10pt}
}
\end{frame}

%7/8 - Results----------------------------------------------
\section{Results}
\begin{frame}{Decomposition Results (\%)}
		\textbf{Males}
    	% \import{beamer/}{male_pct}
		\textbf{Females}
		% \import{beamer/}{female_pct}
\end{frame}

%9/10 - Explained Diffs----------------------------------------------
\begin{frame}{\enquote*{Explained} Differences}
\vspace{-10pt}
    \begin{columns}
    \column{0.56\textwidth}
        \vspace{5pt}
		% \includegraphics[width=\textwidth]{beamer/male_explained_pic}
	\column{0.44\textwidth}
	    \vspace{5pt}
		% \includegraphics[width=\textwidth]{beamer/female_explained_pic.png}
	\end{columns}
	\vspace{-2pt}
	\begin{itemize}
	    \item Occupational clustering
	    \item All have higher levels of education - adds to the literature
	\end{itemize}
\end{frame}

%11 - Trends----------------------------------------------
\begin{frame}{Trends in the \enquote*{Unexplained} Component}
    \begin{columns}
    \column{0.41\textwidth}
        \vspace{5pt}
        \textbf{Males}
		% \import{beamer/}{unexplained_male}
	\column{0.59\textwidth}
	    \vspace{-2pt}
	    \textbf{Females}
		% \import{beamer/}{unexplained_female}
	\end{columns}
	\begin{itemize}
	    \item Slight deterioration since financial crisis
	    \item Consistently worse outcomes for males
	\end{itemize}
\end{frame}

%11 - Endogeniety----------------------------------------------
\begin{frame}{Endogeniety of Occupation}
    \begin{columns}
    \column{0.55\textwidth}
        % \includegraphics[width=1.1\textwidth]{beamer/cross-correls.png}
	\column{0.45\textwidth}
	    \hspace{-20pt}
	    \begin{itemize}
	    \vspace{-20pt}
	    \item Workers are theoretically rewarded for working in undesirable industries
	    \item Not seen in (a), implying clustering is not due to preferences
	    \item Racism does not explain wage premium (b) nor clustering (c)
	    \item Contradiction, possibly due to a poor proxy or industry data
	    \end{itemize}
	\end{columns}
	
\end{frame}

%11 - Conclusion----------------------------------------------
\section{Conclusion}
\begin{frame}{Conclusion}
	 \begin{itemize}
	    \vspace{-20pt}
	    \item Added to literature through dataset and analysis over time
	    \item Increased in wage gaps
	    \item No evidence of direct pay discrimination for women
	    \item Very little change in unexplained wage gaps over time
	    \item Occupational clustering possibly evidence of employer perceptions of quality
	 \end{itemize}
\end{frame}

%11 - References----------------------------------------------
\begin{frame}[allowframebreaks]{Bibliography}
\footnotesize{
Berthoud, R. (2000): `Ethnic employment penalties in Britain,' \textit{Ethnic and Migration
Studies, 26:3, 389-416, DOI: 10.1080/713680490}.
\bigskip
\\Blackaby, D., D. Leslie, P. Murphy, and N. O'Leary (2005): `Born in Britain: How are native ethnic minorities faring in the British labour market?' \textit{Economics Letters, 88, 370-375}.
\bigskip
\\Brynin, M. and A. Guveli (2012): `Understanding the ethnic pay gap in Britain,' \textit{Work, employment and society 26(4) 574-587 DOI: 10.1177/0950017012445095}.
\bigskip
\\GOV.UK (2017): \enquote*{Race in the workplace} \textit{The McGregor Smith review,} \url{https://assets.publishing.service.gov.uk/government/uploads/system/uploads/attachment_data/file/594336/race-in-workplace-mcgregor-smith-review.pdf}, accessed on 18/03/2021
\bigskip
\\Heath, A. and S. Y. Cheung (2006): `Ethnic penalties in the labour market: Employers and discrimination,' \textit{Department for Work and Pensions Research Report No 341.}
\bigskip
\\Leslie, D. and S. Drinkwatr (1999): `Staying on in Full-Time Education: Reasons for Higher Participation Rates Among Ethnic Minority Males and Females,' \textit{Economica 66:63-77, \url{https://doi.org/10.1111/1468-0335.00156}}.
\bigskip
\\Metcalf, H. (2009): `Pay gaps across the equality strands: a review,' \textit{National Institute of Economic and Social Research, Research report: 14}.
\vspace{5pt}
\\ONS (2018): `Ethnicity pay gaps in Great Britain: 2018,' \textit{\url{https://www.ons.gov.uk/employmentandlabourmarket/peopleinwork/earningsandworkinghours/articles/ethnicitypaygapsingreatbritain/2018}}, accessed on 27/02/2021.
}
\end{frame}
\end{document}