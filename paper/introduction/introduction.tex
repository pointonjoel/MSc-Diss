\chapter{Introduction}\label{ch:introduction}

% An introduction in which you set out your objectives and briefly introduce the contents of the following chapters to give the reader an overview of what to expect.

% Setting out the aims and objectives of your project, explaining the overall intention of the project and specific steps that will be taken to achieve that intention.

% Motivation and justification are connected with the knowledge and understanding of the context of the project, and also reflect the ability to make good arguments and think critically. A good motivation would be identifying a research gap or unsolved problem; doing something as an exercise to learn some technique or a programming platform is a bad motivation. A good justification for the methodology will be arguing that some technique has never been used to solve the problem, and giving reasons why it may work well for solving it. 

\section{Motivation} %Explaining the problem being solved.
\label{sec:intro_motivation}

In late 2022, ChatGPT was released, and shocked many users with the quality of human-like responses and vast number of possible applications. The mass use of generative chatbots represents a new phase in \acrfull{ai}. Rather than simply regurgitating facts from a knowledge base, chatbots can now learn from a corpus of text and generate their own human-like output.

A chatbot is an \acrshort{ai} model which attempts to simulate human speech in a written manner, without a physical presence \citep{Nee2023ExploringTT}. The increased availability of high-quality, ``excellent prose" generative chatbots has several benefits \citep{Floridi}, particularly in education, which this \papertype will focus on. These include literature review assistance, data analysis, summarisation of documents, and question answering \citep{lund2023chatting}. (ChatGPT was even used to assist in the generation of some BibTeX citations for this \papertype, to speed up referencing.) However, there are several drawbacks, particularly surrounding the perpetuation of bias, privacy breaches, intellectual property rights and environmental costs associated with training a \acrfull{llm} \citep{Jungherr, Bender21}. When it comes to education specifically, generative chatbots pose an elevated risk of students cheating in assessments, as well as the inability and unreliability of current methods (and even ChatGPT itself) to detect AI-generated text \citep{susnjak2022, Cotton}.

The increased use of \acrshort{ai} within education is ``one of the most important contents of future education development” \citep{BrowMcCoReev2020eu} and is likely to continue to be for many years \citep{zawacki2019systematic}. While there is a need to improve current technology and mitigate the risks and any negative impacts, this \papertype seeks to focus on the opportunities available to education providers. Historically, \acrshort{qa} in the classroom has been difficult and (in higher education) is often constrained to `office hours' or `consultation hours' where lecturers can spend 1-1 time with students to assist with specific queries. However, this can mean that academics repetitively answer similar questions posed by different students. The duplication of questions can be assisted by using platforms such as \citet{Piazza}, that can enable students to see all queries and post their own anonymously. However, this can also lead to duplicated questions as previous questions and answers are not always seen. Additionally, the professor may need to spend a significant amount of time typing up specific answers. Consequently, there is scope for the learning experience of students to be improved.

There are numerous benefits that chatbots can bring to the classroom \citep{Stanislav}. They can provide answers to \acrshort{faq}s and explain difficult concepts in plain language, which makes them popular among students. Furthermore, students can obtain answers almost instantaneously and at any hour of the day, rather than waiting hours or days for a response from an academic. Furthermore, students' research can be streamlined by pointing them to high-quality sources and finding the most relevant parts, particularly where they may have been overwhelmed by the quantity of information available on search engines \citep{Chen22}. For academics, chatbots can answer the \acrshort{faq}s and more simple, repetitive questions that students have. This can free up academics so that they can spend their time more efficiently, to ``reinforce the learning of students", improve their teaching methods \citep{Prez2020}, and ``focus on new experimental designs" \citep{Eva}.

However, generative chatbots such as ChatGPT can produce errors \citep{marcus2018, Bender21, Eva}, particularly when seeking to summarise and abstract documents \citep{Durmus_2020}, and can perpetuate any bias in the training data \citep{geva2019, brown2020}. This means that students could learn inaccurate or irrelevant information, particularly when answers sound plausible and confident but are inaccurate. ChatGPT itself, when asked about the drawbacks of chatbots in education, said that ``Chatbots may lack accountability for errors or malfunctions, which could result in incorrect or harmful guidance to students" \citep{chatgpt23}. Additionally, the most advanced chatbots often lack knowledge after a specific date as the training process is resource-intensive and very time-consuming \citep{Jungherr}.

\section{Aims and Objectives} %Aims and Objectives here.
\label{sec:intro_aims_and_objectives}

The objective of this \papertype is to contribute to the field of long-form, closed-domain generative question answering. The aim is to discover whether this can be accomplished by fine-tuning a pre-trained Transformer model. Additionally, the following chapters will outline which fine-tuning method leads to the best outcomes, and how these methods compare with the use of existing fine-tuned models. Achieving these research aims will enable this \papertype to propose an efficient framework for academics to create a domain-specific, knowledgeable \acrshort{qa} chatbot, where academics can have high confidence in the accuracy of any answers. Importantly, it will seek to be a ``supplement to teaching and learning," rather than replace current teaching methods entirely \citep{Nee2023ExploringTT}. 

%This has been attempted by \citep{Chen22}, using a Deep \acrlong{cnn} (an \acrshort{lstm}-\acrshort{cnn} model). However, a comparison of different models, their training times and practical effectiveness has not been discussed.


% \textcolor{red}{ADD RESEARCH QUESTIONS}

%What are my objectives?

\section{Structure}
\label{sec:intro_paper_structure}
%Explaining what your project is meant to achieve, how it is meant to function, perhaps even a functional specification.

This \papertype will begin by outlining the development of chatbots and the latest state-of-the-art technology used. After this, three potential models will be chosen and justified, the results of their performance will be discussed, and the best-performing model will be outlined. Finally, the findings will be summarised and complemented by a discussion of the recommended direction of future research. All code is available in the following GitHub repository: \url{https://github.com/pointonjoel/MSc-Diss}. %against which benchmark?? 

% \textcolor{red}{It's currently being trialed by Khan Academy https://blog.khanacademy.org/harnessing-ai-so-that-all-students-benefit-a-nonprofit-approach-for-equal-access/}